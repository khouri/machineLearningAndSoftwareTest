\section{Metamorphic}

\begin{frame}
	\begin{block}{Artigo 04}
	\begin{enumerate}
		\item Identifying Implementation Bugs in Machine Learning Based Image Classifiers using Metamorphic Testing \cite{Metamorphic}
	\end{enumerate}
	\end{block}
\end{frame}

\begin{frame}
	\begin{block}{Metamorphic - overview}
	\begin{enumerate}
		\item Uso do conceito de teste metamórfico para testar algoritmos de machine learning.
		\item Solução não agnóstica dado que há necessidade de definir relação metamórfica para cada novo modelo
		\item Os autores conseguem validar a solução criando bugs artificiais em algoritmos conhecidos de machine learning (como SVM) e aplicando a solução proposta.
		\item Não há uso de bugs reais para avaliar o algoritmo
	\end{enumerate}
	\end{block}
\end{frame}

\begin{frame}
	\begin{block}{Metamorphic - funcionamento}
		\begin{enumerate}
			\item São definidos relações metamórficas para cada um dos dois algoritmos de machine learning (SVM e Residual Network)
			\item Os autores alteram o fonte dos algoritmos para chumbar uma semente aleatória fixa para todas as execuções
			\item As relações metamórficas não devem causar alteração no output e  função de perda no decorrer do treino
			\item Com essas condições asseguradas, se ocorrer uma alteração na função de perda ou output há indício de bug
		\end{enumerate}
	\end{block}
\end{frame}


\begin{frame}
	\begin{block}{Metamorphic - exemplos de relações metamórficas}
		\begin{enumerate}
			\item Alterar a ordem das instâncias de treino
			\item Alterar a ordem das features de treino
			\item Adicionar um valor constantes em determinadas features
			\item Escalar as variáveis
		\end{enumerate}
	\end{block}
\end{frame}
